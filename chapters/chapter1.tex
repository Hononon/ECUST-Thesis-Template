\chapter{模板示例}\label{chapter1}

\section{写作说明}
本章用于展示图、表、算法与公式的基本用法。正文内容请根据实际论文进行替换,引用示例见文献\upcite{latexcompanion,lamport1994latex}。

\section{图表示例}
\subsection{插图}
插图见图\ref{fig:template-figure}。
\begin{figure}[H]
  \centering
  \includegraphics[width=0.2\textwidth]{figures/chapter1/fig.png}
  \bicaption{模板示例插图}{Sample figure in template}
  \label{fig:template-figure}
\end{figure}

\subsection{表格}
\begin{table}[H]
  \centering
  \bicaption{模板示例表格}{Sample table in template}
  \label{tab:template-table}
  \begin{tabular}{ccc}
    \hline
    项目 & 数值 & 备注 \\
    \hline
    A  & 1  & 示例 \\
    B  & 2  & 示例 \\
    \hline
  \end{tabular}
\end{table}

\section{算法示例}
\begin{algorithm}[H]
  \caption{示例算法}
  \label{alg:template}
  \begin{algorithmic}[1]
    \Require 输入向量 $\mathbf{x}$
    \Ensure 输出结果 $y$
    \State $y \gets 0$
    \For{each $x_i$ in $\mathbf{x}$}
    \State $y \gets y + x_i$
    \EndFor
    \State \Return $y$
  \end{algorithmic}
\end{algorithm}

\section{公式示例}
行内公式示例:$E=mc^2$。

块级公式示例:
\begin{equation}
  y = \sum_{i=1}^{n} x_i
  \label{eq:template}
\end{equation}

\section{定理等环境示例}
\begin{remark}
  这是一个示例备注环境。
\end{remark}

\begin{theorem}
  这是一个示例定理环境。
\end{theorem}
\begin{proof}
  这是定理的证明过程示例。
\end{proof}

\begin{assumption}
  这是一个示例假设环境。
\end{assumption}
